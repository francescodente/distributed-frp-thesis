\begin{abstract}
The challenges faced by developers when dealing with large-scale, distributed systems call for high level abstractions in order to manage their growing complexity.
%
In this context, aggregate computing is an emerging paradigm that allows to declaratively specify the behavior of these types of systems, by viewing the system as a whole, rather than focussing on the interactions of individual devices.
%
This new way of reasoning about the behavior of aggregates of devices is based on a formal calculus that describes programs as the functional composition of computational fields that evolve through time, called \quotes{field calculus}.
%
As it currently stands, though, this formal model does not provide native ways to fine-tune the timing and the evolution dynamics of fields, which is desirable to avoid wasteful usage of processing resources.

In this paper we propose to combine aggregate computing with the functional reactive programming to develop an evolution of the current model proposed by field calculus that allows to effectively specify aggregate computations as reactions to changes in the environment.
%
A small prototype has been developed to assess the applicability of the proposed model and to verify that some of the properties of the original model remain observable in the new one.
%
Ultimately, this project aims to contribute towards the vision of functional reactive self-organization.
\end{abstract}
