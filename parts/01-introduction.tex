\chapter{\introductionname}

The complexities that developers are faced with nowadays when dealing with large-scale, distributed systems have grown to a point that imposes the adoption of new techniques to manage those complexities and to help reasoning about distributed software at high levels of abstraction.
%
Typically, systems of this nature fall under the category of \textit{Collective Adaptive Systems}, in which a large number of devices pursues a global goal by means of \textit{strongly decentralized} interactions, while adapting its behavior to the constantly changing environment.
%
New paradigms should allow to \textit{declaratively} specify the behavior of such types of system, while also providing \textit{composable} and \textit{reusable} building blocks.

One of the most promising approaches in this matter is \textit{aggregate computing}, which focuses on the definition of behaviors of aggregates of devices, rather than thinking in terms of single entities.
%
This leads to the idea of \textit{self-organization}, thanks to which global coordination behavior \textit{emerges} from local coordination abstractions.
%
Its foundation is given by \textit{field calculus}, a formal calculus that defines aggregate programs as the functional composition of \textit{computational fields}.

A set of frameworks and languages based on this paradigm already exist, both as internal or external Domain Specific Languages, like ScaFi or Protelis.
%
However, these frameworks -- and field calculus as well -- propose a \textit{proactive model} based on computation rounds, which lacks abstractions that naturally model the behavior of a system as a reaction to relevant events in the surrounding environment.
%
This means that it is not currently possible to fine-tune the overall dynamics and the timing of computations, resulting in wasteful usage of the processing resources.

The objective of this thesis is to explore the issues of the proactive model, proposing a new one based on \textit{reactivity} to environmental changes, leveraging the principles of \textit{functional reactive programming} to keep the compositionality aspects intact and to improve on some of the current shortcomings.
%
On top of that, a small, but functional prototype will be developed for said model, both to assess its feasibility and to be aware of some of the caveats that a production-ready implementation should tackle.
%
Ultimately, the vision of the whole project is to take a step towards \textit{functional reactive self-organization}.

\paragraph{Thesis structure}

The remainder of this thesis is organized as follows.
%
\Cref{chap:background} introduces the essential background on all the subjects upon which all the work is based, presenting concepts of \textit{functional programming}, the new features of \textit{Scala 3}, the basics of \textit{aggregate computing} and \textit{field calculus}, and finally an overview of the \textit{functional reactive programming} paradigm.
%
On this foundation, \Cref{chap:analysis-design} analyzes the current state of aggregate computing, in particular with the \textit{ScaFi} framework, and proposes a specification for a reactive model that .
%
Subsequently, \Cref{chap:implementation} goes into the details of how the prototype for the specification was implemented with Scala 3.
%
\Cref{chap:evaluation} shows the process that was adopted in order to evaluate both the model and the implementation, through unit tests and empirical tests.
%
Finally, \Cref{chap:conclusions} gives some final thoughts on the contribution of the thesis and paves the way for future work.